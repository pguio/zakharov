%
% $Id: damping.tex,v 1.6 2007/06/13 16:07:39 patrick Exp $
%
% Copyright (c) 2000 Patrick Guio <patrick@phys.uit.no>
%
% All Rights Reserved.
%
% This program is free software; you can redistribute it and/or modify it
% under the terms of the GNU General Public License as published by the
% Free Software Foundation; either version 2.  of the License, or (at your
% option) any later version.
%
% This program is distributed in the hope that it will be useful, but
% WITHOUT ANY WARRANTY; without even the implied warranty of
% MERCHANTABILITY or FITNESS FOR A PARTICULAR PURPOSE.  See the GNU General
% Public License for more details.
%

\documentclass[10pt,a4paper]{article}
\usepackage[T1]{fontenc}
\usepackage{times}
%\usepackage{mathpple}
\usepackage{textcomp}
\usepackage{a4wide}
\usepackage{amsmath}
\usepackage[italic]{esdiff}
\usepackage{physics}
\usepackage{braket}
\usepackage{html}
\usepackage{graphicx}
\usepackage{color}
\pagecolor{white}

\parindent 0cm

\newlength{\mylength}

\newenvironment{feqnarray}%
{\setlength{\fboxsep}{15pt}
\setlength{\fboxrule}{0.5pt}
\setlength{\mylength}{\textwidth}
\addtolength{\mylength}{-2\fboxsep}
\addtolength{\mylength}{-2\fboxrule}
\Sbox
\minipage{\mylength}%
	\setlength{\abovedisplayskip}{-2\lineskip}
	\setlength{\belowdisplayskip}{-2\lineskip}
\begin{eqnarray}}%
{\end{eqnarray}\endminipage\endSbox
\[\fbox{\TheSbox}\]}


\def\Eq#1{Eq.~(#1)}
\def\Eqs#1{Eqs.~(#1)}

\def\matlab/{\textsc{Matlab}}

\bibliographystyle{egs-ag}

\title{Langmuir damping}
\author{Patrick Guio}
\date{\normalsize$ $Date: 2007/06/13 16:07:39 $ $,~ $ $Revision: 1.6 $ $}

\begin{document}

\maketitle

\section{Definition}
The dispersion relation for Langmuir wave is written
\begin{align*}
\omega_r^2&=\omega_e^2\left(1{+}3\alpha^2\right)\\
\omega_i&=\frac{\pi}{2}\frac{\omega_e^2}{k^2}\omega_r
\diff*{f}{v}{\omega_r/k}
\end{align*}
where $\alpha=k\lambda_e$ is the wave number normalised to the electron 
Debye length $\lambda_e=v_e/w_e$, $v_e$ is the thermal velocity at temperature
$T_e$ and $\omega_e$ is the electron plasma frequency.
$v_\phi=\omega_r/k$ is the phase velocity of the Langmuir wave with
frequency $\omega_r$ and wave number $k$. 
$f$ is the electron velocity
distribution function.

If we consider a Maxwellian velocity distribution function with 
mean velocity $u$ and thermal velocity $\theta$
\begin{equation}
f(v)=\frac{1}{\sqrt{2\pi}}\frac{1}{\theta}
\exp\left(-\frac{1}{2}\left(\frac{v-u}{\theta}\right)^2\right)
\end{equation}
the Langmuir damping is written
\begin{equation}
\omega_i=-\sqrt{\frac{\pi}{8}}\frac{\omega_e^2}{k^2\theta^2}\frac{|k|}{k}
\omega_r\frac{\omega_r-ku}{k\theta}
\exp\left(-\frac{1}{2}\left(\frac{\omega_r-ku}{k\theta}\right)^2\right)
\end{equation}
Now normalising to the thermal velocity $v_e$ leads to
\begin{equation}
\omega_i=-\sqrt{\frac{\pi}{8}}\frac{\omega_e}{\alpha^2}\frac{|k|}{k}
\frac{v_e^2}{\theta^2}\sqrt{1{+}3\alpha^2}
\left(
\frac{\sqrt{1{+}3\alpha^2}}{\alpha}\frac{v_e}{\theta}-\frac{u}{\theta}
\right)
\exp\left(-\frac{1}{2}\left(
\frac{\sqrt{1{+}3\alpha^2}}{\alpha}\frac{v_e}{\theta}-\frac{u}{\theta}
\right)^2\right)
\end{equation}


The Zakharov normalised variables  $\tilde{k}$ and $\tilde{t}$ are related to
their physical counterparts by
\begin{align*}
\tilde{k}&=\frac{3}{2}\sqrt{\frac{m_i}{m_e\eta}}\frac{v_e}{\omega_e}k\\
\tilde{\omega}&=\frac{3}{2}\frac{m_i}{m_e\eta}\frac{\omega}{\omega_e}
\end{align*}

The damping is thus written
\begin{equation}
\tilde{\omega}_i=-\sqrt{\frac{\pi}{8}}
\left(\frac{3}{2}\right)^3\left(\frac{m_i}{m_e\eta}\right)^2
\frac{1}{\tilde{k}^2}\frac{|\tilde{k}|}{\tilde{k}}\frac{v_e^2}{\theta^2}
\sqrt{1{+}3\alpha^2}
\left(
\frac{\sqrt{1{+}3\alpha^2}}{\alpha}\frac{v_e}{\theta}-\frac{u}{\theta}
\right)
\exp\left(-\frac{1}{2}\left(
\frac{\sqrt{1{+}3\alpha^2}}{\alpha}\frac{v_e}{\theta}-\frac{u}{\theta}
\right)^2\right)
\end{equation}
where $\alpha$ can be expressed in terms of the normalised wave number
\[
\alpha = \tilde{k}\frac{2}{3}\sqrt{\frac{m_e\eta}{m_i}}
\]

\section{Background electrons}
The background electrons with $u=0$ and $\theta=v_e$ give
\begin{align}
\omega_i&=-\sqrt{\frac{\pi}{8}}\frac{\omega_e}{\alpha^2|\alpha|}
\left(1{+}3\alpha^2\right)\exp\left(-\frac{1}{2\alpha^2}-\frac{3}{2}\right)\\
\tilde{\omega}_i&=-\sqrt{\frac{\pi}{8}}
\left(\frac{3}{2}\right)^4\left(\frac{m_i}{m_e\eta}\right)^{5/2}
\frac{1}{\tilde{k}^2|\tilde{k}|}
\left(1+\frac{4}{3}\tilde{k}^2\frac{m_e\eta}{m_i}\right)
\exp\left(-\frac{9}{8}\frac{m_i}{m_e\eta}\frac{1}{\tilde{k}^2}-
\frac{3}{2}\right)
\end{align}


\section{Beam}
For a beam with parameters $u=v_b$ and $\theta=\Delta v_b$, and posing 
$\beta=\sqrt{1{+}3\alpha^2}$ the Landau damping is written
\begin{align}
\omega_i&=-\sqrt{\frac{\pi}{8}}\frac{\omega_e}{\alpha^2}\frac{|k|}{k}
\frac{v_e^2}{\Delta v_b^2}\beta
\left(\frac{\beta}{\alpha}\frac{v_e}{\Delta v_b}-
\frac{v_b}{\Delta v_b}\right)
\exp\left(-\frac{1}{2}\left(
\frac{\beta}{\alpha}\frac{v_e}{\Delta v_b}-\frac{v_b}{\Delta v_b}
\right)^2\right)\\
\tilde{\omega}_i&=-\sqrt{\frac{\pi}{8}}
\left(\frac{3}{2}\right)^3\left(\frac{m_i}{m_e\eta}\right)^2
\frac{1}{\tilde{k}^2}\frac{|\tilde{k}|}{\tilde{k}}
\frac{v_e^2}{\Delta v_b^2}\beta
\left(\frac{\beta}{\alpha}\frac{v_e}{\Delta v_b}-
\frac{v_b}{\Delta v_b}\right)
\exp\left(-\frac{1}{2}\left(
\frac{\beta}{\alpha}\frac{v_e}{\Delta v_b}-\frac{v_b}{\Delta v_b}
\right)^2\right)
\end{align}


\end{document}
