%
% $Id: payne_1983.tex,v 1.6 2005/04/19 10:59:44 patrickg Exp $
%
% Copyright (c) 2000 Patrick Guio <patrick@phys.uit.no>
%
% All Rights Reserved.
%
% This program is free software; you can redistribute it and/or modify it
% under the terms of the GNU General Public License as published by the
% Free Software Foundation; either version 2.  of the License, or (at your
% option) any later version.
%
% This program is distributed in the hope that it will be useful, but
% WITHOUT ANY WARRANTY; without even the implied warranty of
% MERCHANTABILITY or FITNESS FOR A PARTICULAR PURPOSE.  See the GNU General
% Public License for more details.
%

\documentclass[10pt,a4paper,twocolumn]{article}
\usepackage{a4wide}
\usepackage{times}
\usepackage{mathpple}
\usepackage[T1]{fontenc}
\usepackage{epsf}
\usepackage{natbib}

\parindent 0cm

\def\Eq#1{Eq.~(#1)}

\def\matlab/{\textsc{Matlab}}

%\renewcommand{\floatpagefraction}{.9}
%\renewcommand{\textfraction}{.1}
%\renewcommand{\topfraction}{.8}
%\renewcommand{\bottomfraction}{.8}

%\def\tablesize{\footnotesize}
\def\tablesize{\small}

\bibliographystyle{egs-ag}

\title{Computed figures and tables for the 1D periodic Zakharov solver
described in \\ \cite{payne:1983}}
\author{Patrick Guio}
\date{\normalsize$ $Date: 2005/04/19 10:59:44 $ $,~ $ $Revision: 1.6 $ $}

\begin{document}

\maketitle

\tableofcontents

\section{Introduction}

This document contains the figures and tables in \cite{payne:1983}.

\section{Zakharov equations}

The Zakharov system of equations is written
\begin{eqnarray}
i{\partial E\over\partial t}+{\partial^2 E\over\partial x^2} &=& nE\\
{\partial^2 n\over\partial t}-{\partial^2 n\over\partial x^2} &=& 
{\partial^2 |E|^2\over\partial x^2}
\end{eqnarray}

\section{Numerical results}

\begin{table}[ht]
\centerline{\tablesize \begin{tabular}{lcccccc}
\hline\hline 
\csname @@input\endcsname table1.dat
\hline\hline
\end{tabular}}
\caption{}
\label{table1}
\end{table}

\begin{figure}[ht]
\centerline{\setlength{\epsfxsize}{.75\columnwidth}\epsffile{fig1.eps}}
\caption{}
\end{figure}

\begin{figure}[ht]
\centerline{\setlength{\epsfxsize}{.75\columnwidth}\epsffile{fig2.eps}}
\caption{}
\end{figure}

\begin{table}[ht]
\centerline{\tablesize \begin{tabular}{lcccccc}
\hline\hline 
\csname @@input\endcsname table2.dat
\hline\hline
\end{tabular}}
\caption{}
\label{table2}
\end{table}

\begin{table}[ht]
\centerline{\tablesize \begin{tabular}{ccccc}
\hline\hline 
\csname @@input\endcsname table3.dat
\hline\hline
\end{tabular}}
\caption{}
\label{table3}
\end{table}

\begin{figure}[ht]
\centerline{\setlength{\epsfxsize}{.75\columnwidth}\epsffile{fig3.eps}}
\caption{}
\end{figure}

\begin{table}[ht]
\centerline{\tablesize \begin{tabular}{lcccccc}
\hline\hline 
\csname @@input\endcsname table4.dat
\hline\hline
\end{tabular}}
\caption{}
\label{table4}
\end{table}

\begin{figure}[ht]
\centerline{\setlength{\epsfxsize}{.75\columnwidth}\epsffile{fig4.eps}}
\caption{}
\end{figure}

\begin{figure}[ht]
\centerline{\setlength{\epsfxsize}{.75\columnwidth}\epsffile{fig5.eps}}
\caption{}
\end{figure}

\setcounter{figure}{6}
\begin{figure}[ht]
\centerline{\setlength{\epsfxsize}{.75\columnwidth}\epsffile{fig7.eps}}
\caption{}
\end{figure}

\begin{figure}[ht]
\centerline{\setlength{\epsfxsize}{.75\columnwidth}\epsffile{fig8.eps}}
\caption{}
\end{figure}

\begin{figure}[ht]
\centerline{\setlength{\epsfxsize}{.75\columnwidth}\epsffile{fig9.eps}}
\caption{}
\end{figure}

\begin{figure}[ht]
\centerline{\setlength{\epsfxsize}{.75\columnwidth}\epsffile{fig10.eps}}
\caption{}
\end{figure}

\begin{figure}[ht]
\centerline{\setlength{\epsfxsize}{.75\columnwidth}\epsffile{fig11.eps}}
\caption{}
\end{figure}

\begin{figure}[ht]
\centerline{\setlength{\epsfxsize}{.75\columnwidth}\epsffile{fig12.eps}}
\caption{}
\end{figure}

\bibliography{research}

\end{document}
