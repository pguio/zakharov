%
% $Id: zakharov_sources.tex,v 1.55 2011/03/31 18:48:28 patrick Exp $
%
% Copyright (c) 2000 Patrick Guio <patrick@phys.uit.no>
%
% All Rights Reserved.
%
% This program is free software; you can redistribute it and/or modify it
% under the terms of the GNU General Public License as published by the
% Free Software Foundation; either version 2.  of the License, or (at your
% option) any later version.
%
% This program is distributed in the hope that it will be useful, but
% WITHOUT ANY WARRANTY; without even the implied warranty of
% MERCHANTABILITY or FITNESS FOR A PARTICULAR PURPOSE.  See the GNU General
% Public License for more details.
%

\documentclass[10pt,a4paper]{article}
\usepackage[latin1,utf8]{inputenc}
\usepackage[empty]{fullpage}
%\usepackage{a4wide}
\usepackage{times}
%\usepackage{mathpple}
\usepackage{amsmath}
\usepackage[italic]{esdiff}
\usepackage{physics}
\usepackage{graphicx}
\usepackage{fancybox}
\usepackage{natbib}

\parindent 0cm

\newcommand{\eq}[1]{Eq.~(\ref{#1})}
\newcommand{\eqs}[2]{Eqs.~(\ref{#1}--\ref{#2})}

\newlength{\mylength}

\newenvironment{falign}%
{\setlength{\fboxsep}{15pt}
\setlength{\fboxrule}{0.5pt}
\setlength{\mylength}{\textwidth}
\addtolength{\mylength}{-2\fboxsep}
\addtolength{\mylength}{-2\fboxrule}
\Sbox
\minipage{\mylength}%
	\setlength{\abovedisplayskip}{-2\lineskip}
	\setlength{\belowdisplayskip}{-2\lineskip}
\align}%
{\endalign\endminipage\endSbox
\[\fbox{\TheSbox}\]}


\def\matlab/{\textsc{Matlab}}


%\def\tablesize{\footnotesize}
\def\tablesize{\small}

\def\knui{\sqrt{k^2{-}\nu_i^2}}
\def\knui{\left(k^2{-}\nu_i^2\right)^\frac{1}{2}}

\def\dn{\delta n}
\def\dnk{\delta n_k}
\def\dnj{\delta n_j}
\def\dE{\delta E}
\def\dEk{\delta E_k}
\def\dEj{\delta E_j}

\bibliographystyle{egs-ag}

\title{One-dimensional periodic Zakharov solver with sources}
\author{Patrick Guio}
\date{\normalsize$ $Date: 2011/03/31 18:48:28 $ $,~ $ $Revision: 1.55 $ $}

\begin{document}

\maketitle

\section{Zakharov equations}
\begin{subequations}
The Zakharov system of equations is written in physical SI units
\begin{falign}
\divg
\left(i\diffp{}{t}+i\nu_e\times+\frac{3}{2}\omega_e\lambda_e^2\lapl\right)
\vec{E} &= \divg\left(\frac{\omega_e}{2}\frac{\delta n}{\ne}\vec{E}\right),
\label{eq:pzake}\\
\left(\diffp[2]{}{t}+2\nu_i\times\diffp{}{t}-c_s^2\lapl\right)\delta n &=
\frac{\epso}{4m_i}\lapl\left|\vec{E}\right|^2,
\label{eq:pzakn}
\end{falign}
\end{subequations}
where $\cross$ denotes the convolution product operator, $\omega_e=(n_ee^2/m_e\epso)^{1/2}$ 
is the electron plasma frequency, $v_e=(\kB\Te/\me)^{1/2}$ the electron thermal velocity, 
$\lambda_e=v_e/\omega_e$ the Debye length and $c_s$ the
ion acoustic velocity related to $v_e$ through
\begin{align*}
c_s^2 = \frac{\eta m_e}{m_i}v_e^2, \qquad\text{where}\qquad
\eta = \frac{\Te+3T_i}{\Te},
\end{align*}
and $\nu_e$, $\nu_i$ are the electron and ion linear momentum transfer
collision frequencies respectively including kinetically determined linear
Landau damping of Langmuir waves and ion-acoustic waves 
\begin{align*}
\nu_e = \frac{\nu_{ec}}{2}+\gamma_L, \qquad\text{and}\qquad
\nu_i = \frac{\nu_{ic}}{2}+\gamma_{ia}.
\end{align*}

With the normalised variables
\begin{align*}
\tilde{t} &= \frac{t}{\tau},&
\tilde{\vec{r}} &= \frac{\vec{r}}{\chi},&
\tilde{n} &= \frac{n}{\nu},&
\tilde{E} &= \frac{E}{\varepsilon}\\
\tau &= \frac{3}{2}\frac{m_i}{\eta m_e}\frac{1}{\omega_e}, &
\chi& = \frac{3}{2}\sqrt{\frac{m_i}{\eta m_e}} \lambda_e, &
\nu &= \frac{4}{3}\frac{\eta m_e}{m_i}n_e, &
\varepsilon &= \frac{4\eta}{\sqrt{3}}\sqrt{\frac{m_e}{m_i}}
\sqrt{\frac{n_e\kB\Te}{\epso}}
\end{align*}
and scaling factors
\begin{align*}
\tilde{t} &= \left(\frac{2}{3}\frac{\eta m_e}{m_i}\omega_e\right)t, &
\tilde{\vec{r}} &= \left(\frac{2}{3}\sqrt{\frac{\eta m_e}{m_i}}
\frac{1}{\lambda_e}\right)\vec{r}, \\
\tilde{n} &= \left(\frac{3}{4}\frac{m_i}{\eta m_e}\frac{1}{n_e}\right)
\delta n, &
\tilde{E} &= \left(\frac{\sqrt{3}}{4\eta}\sqrt{\frac{m_i}{m_e}}
\sqrt{\frac{\epso}{n_e\kB\Te}}\right)E,
\end{align*}
the system of equations is rewritten in one-dimensional space
\begin{align}
\left(i\diffp{}{{\tilde{t}}}+i\tilde{\nu}_e\times+\diffp[2]{}{\tilde{x}}\right)
\tilde{E} &= \tilde{n}\tilde{E}
\label{eq:nzake}\\
\left(\diffp[2]{}{\tilde{t}}+2\tilde{\nu}_i\times\diffp{}{{\tilde{t}}}-
\diffp[2]{}{\tilde{x}}\right)\tilde{n} &=
\diffp[2]{|\tilde{E}|^2}{\tilde{x}}
\label{eq:nzakn}
\end{align}


\section{Pseudo spectral integration method}

Assuming a one-dimensional periodic space, we take the space Fourier
transform of the Zakharov system of equations and get
\begin{align}
\diff{E_k}{t}+\nu_eE_k+ik^2E_k &= -i\left(nE\right)_k
\label{eq:zake}\\
\diff[2]{n_k}{t}+2\nu_i\diff{n_k}{t}+k^2n_k &= -k^2\left|E\right|^2_k
\label{eq:zakn}
\end{align}

\eqs{eq:zake}{eq:zakn} can be written in their integral form
\begin{align}
E_k(t) & = E_k(0)\exp\left(-(ik^2{+}\nu_e)t\right)-i\int_0^t
\left(nE\right)_k\exp\left(-(ik^2{+}\nu_e)(t{-}s)\right)\dint{s},\\
\begin{split}
n_k(t) & = n_k(0)\exp(-\nu_it)\left(\cos\knui t+\nu_i\frac{\sin\knui t}{\knui}
\right)+\\
&\quad n_k^\prime(0)\exp(-\nu_it)
\frac{\sin\knui t}{\knui}-
\int_0^tk^2\left|E\right|^2_k\exp\left(-\nu_i(t{-}s)\right)
\frac{\sin(\knui(t{-}s))}{\knui}\dint{s}.
\end{split}
\end{align}

Using the trapezoidal method of integration
\begin{align*}
\int_0^hf(x)\dint{x}=h\left(\frac{1}{2}f(0)+\frac{1}{2}f(h)\right)+
\mathcal{O}(h^3f^{\prime\prime}),
\end{align*}
$n_k$ can be integrated one iteration step $h$ ahead 
\begin{falign}
\begin{split}
n_k(h) &= n_k(0)\exp(-\nu_ih)\left(\cos\knui h+
\nu_i\frac{\sin\knui h}{\knui}\right)\\
&\quad n_k^\prime(0)\exp(-\nu_ih)\frac{\sin\knui h}{\knui}
-\frac{h}{2}k^2\left|E(0)\right|^2_k\exp(-\nu_ih)
\end{split}
\end{falign}

In the same way, $E_k$ is integrated one iteration step $h$ ahead
\begin{align*}
E_k(h) = E_k(0)\exp\left(-(ik^2{+}\nu_e)h\right)-
i\frac{h}{2}\left(\left(n(0)E(0)\right)_k\exp\left(-(ik^2{+}\nu_e)h\right)+
\left(n(h)E(h)\right)_k\right)
\end{align*}
Posing
\begin{align*}
\widehat{E}_k = E_k(0)\exp\left(-(ik^2{+}\nu_e)h\right)\quad\text{and}\quad
(\widehat{nE})_k = \left(n(0)E(0)\right)_k\exp\left(-(ik^2{+}\nu_e)h\right)
\intertext{yields}
E_k(h)=\widehat{E}_k-i\frac{h}{2}\left(\widehat{nE})_k+
\left(n(h)E(h)\right)_k\right)
\end{align*}
Then taking the inverse Fourier transform of this equation leads to
\begin{falign}
E_j(h)=\frac{\widehat{E}_j-i\frac{h}{2}(\widehat{nE})_j}{1+i\frac{h}{2}n_j(h)}
\end{falign}

In order to calculate $n_k^\prime$ one step iteration step ahead, let us 
pose $v_k=n_k^\prime$, \eq{eq:zakn} is written
\begin{align*}
v_k^\prime+2\nu_iv_k=-k^2n_k-k^2\left|E\right|^2_k
\end{align*}
The integral form of this equation is
\begin{align*}
v_k(t) = v_k(0)\exp(-2\nu_it)-
k^2\int_0^t\left(n_k+\left|E\right|^2_k\right)\exp(-2\nu_i(t{-}t^\prime))
\dint{t^\prime}
\end{align*}
Applying the trapezoidal rule leads to
\begin{falign}
n^\prime_k(h) = n^\prime_k(0)\exp(-2\nu_ih)-
\frac{h}{2}k^2\left(\left(n_k(0){+}\left|E(0)\right|^2_k\right)
\exp(-2\nu_ih)+n_k(h)+\left|E(h)\right|^2_k\right)
\end{falign}

\section{Ion acoustic damping}
The dispersion relation for ion acoustic wave is written
\begin{align}
\omega_r&=kc_s\\
\omega_i&=\gamma_{ia}=\sqrt{\frac{\pi}{8}}
\left(\left(\frac{m_e}{m_i}\right)^\frac{1}{2}+
\left(\frac{\Te}{T_i}\right)^\frac{3}{2}
\exp\left(-\frac{\Te}{2T_i}-\frac{3}{2}\right)\right)|k|c_s\\
\tilde{\omega}_i&=\sqrt{\frac{\pi}{8}}
\left(\left(\frac{m_e}{m_i}\right)^\frac{1}{2}+
\left(\frac{\Te}{T_i}\right)^\frac{3}{2}
\exp\left(-\frac{\Te}{2T_i}-\frac{3}{2}\right)\right)|\tilde{k}|
\end{align}


\section{Landau damping}
The dispersion relation for Langmuir wave is written
\begin{align}
\omega_r^2&=\omega_e^2\left(1{+}3\alpha^2\right)\\
\omega_i&=\gamma_L=\frac{\pi}{2}\frac{\omega_e^2}{k^2}\omega_r
\diff*{f}{v}{\omega_r/k}
\end{align}
where $\alpha=k\lambda_e$ is the wave number normalised to the electron
Debye length $\lambda_e=v_e/w_e$, $v_e$ is the thermal velocity at
temperature
$\Te$ and $\omega_e$ is the electron plasma frequency.
$v_\phi=\omega_r/k$ is the phase velocity of the Langmuir wave with
frequency $\omega_r$ and wave number $k$.
$f$ is the electron velocity
distribution function.

If we consider a Maxwellian velocity distribution function with
mean velocity $u$ and thermal velocity $\theta$
\begin{equation}
f(v)=\frac{1}{\sqrt{2\pi}}\frac{1}{\theta}
\exp\left(-\frac{1}{2}\left(\frac{v-u}{\theta}\right)^2\right)
\end{equation}
the Langmuir damping is written
\begin{equation}
\omega_i=-\sqrt{\frac{\pi}{8}}\frac{\omega_e^2}{k^2\theta^2}\frac{|k|}{k}
\omega_r\frac{\omega_r-ku}{k\theta}
\exp\left(-\frac{1}{2}\left(\frac{\omega_r-ku}{k\theta}\right)^2\right)
\end{equation}
Now normalising to the thermal velocity $v_e$ and posing 
$\beta^2=1{+}3\alpha^2$ 
leads to
\begin{equation}
\omega_i=-\sqrt{\frac{\pi}{8}}\frac{\omega_e}{\alpha^2}\frac{|k|}{k}
\frac{v_e^2}{\theta^2}\beta
\left(\frac{\beta}{\alpha}\frac{v_e}{\theta}-\frac{u}{\theta}\right)
\exp\left(-\frac{1}{2}\left(
\frac{\beta}{\alpha}\frac{v_e}{\theta}-\frac{u}{\theta}
\right)^2\right)
\end{equation}

The Zakharov normalised variables  $\tilde{k}$ and $\tilde{t}$ are related
to their physical counterparts by
\begin{align*}
\tilde{k}&=\frac{3}{2}\sqrt{\frac{m_i}{m_e\eta}}\frac{v_e}{\omega_e}k\\
\tilde{\omega}&=\frac{3}{2}\frac{m_i}{m_e\eta}\frac{\omega}{\omega_e}
\end{align*}

The damping in normalised units is thus written
\begin{equation}
\tilde{\omega}_i=-\sqrt{\frac{\pi}{8}}
\left(\frac{3}{2}\right)^3\left(\frac{m_i}{m_e\eta}\right)^2
\frac{1}{\tilde{k}^2}\frac{|\tilde{k}|}{\tilde{k}}\frac{v_e^2}{\theta^2}
\beta\left(\frac{\beta}{\alpha}\frac{v_e}{\theta}-\frac{u}{\theta}\right)
\exp\left(-\frac{1}{2}\left(
\frac{\beta}{\alpha}\frac{v_e}{\theta}-\frac{u}{\theta}
\right)^2\right)
\end{equation}
where $\alpha$ can be expressed in terms of the normalised wave number
\begin{align*}
\alpha = k\lambda_e=\tilde{k}\frac{2}{3}\sqrt{\frac{m_e\eta}{m_i}}
\end{align*}

\subsection{Background electrons}
The background electrons with $u=0$ and $\theta=v_e$ give
\begin{align}
\omega_i&=-\sqrt{\frac{\pi}{8}}\frac{\omega_e}{\alpha^2|\alpha|}
\beta\exp\left(-\frac{1}{2\alpha^2}-\frac{3}{2}\right)\\
\tilde{\omega}_i&=-\sqrt{\frac{\pi}{8}}
\left(\frac{3}{2}\right)^4\left(\frac{m_i}{m_e\eta}\right)^{5/2}
\frac{1}{\tilde{k}^2|\tilde{k}|}
\left(1+\frac{4}{3}\tilde{k}^2\frac{m_e\eta}{m_i}\right)
\exp\left(-\frac{9}{8}\frac{m_i}{m_e\eta}\frac{1}{\tilde{k}^2}-
\frac{3}{2}\right)
\end{align}

\subsection{Beam}
For a beam with parameters $u=v_b$ and $\theta=\Delta v_b$,
the Landau damping is written
\begin{align}
\omega_i&=-\sqrt{\frac{\pi}{8}}\frac{\omega_e}{\alpha^2}\frac{|k|}{k}
\frac{v_e^2}{\Delta v_b^2}\beta
\left(\frac{\beta}{\alpha}\frac{v_e}{\Delta v_b}-
\frac{v_b}{\Delta v_b}\right)
\exp\left(-\frac{1}{2}\left(
\frac{\beta}{\alpha}\frac{v_e}{\Delta v_b}-\frac{v_b}{\Delta v_b}
\right)^2\right)\\
\tilde{\omega}_i&=-\sqrt{\frac{\pi}{8}}
\left(\frac{3}{2}\right)^3\left(\frac{m_i}{m_e\eta}\right)^2
\frac{1}{\tilde{k}^2}\frac{|\tilde{k}|}{\tilde{k}}
\frac{v_e^2}{\Delta v_b^2}\beta
\left(\frac{\beta}{\alpha}\frac{v_e}{\Delta v_b}-
\frac{v_b}{\Delta v_b}\right)
\exp\left(-\frac{1}{2}\left(
\frac{\beta}{\alpha}\frac{v_e}{\Delta v_b}-\frac{v_b}{\Delta v_b}
\right)^2\right)
\end{align}

\section{Plasma line}
The Zakharov equations are split into a fast time scale and a slow time
scale equation. To relate the fast electron density response yo the
electrostatic Langmuir envelope field, we use Poisson equation
\begin{align}
\diffp{E_L}{x}=-\frac{e}{\epso}(n_e-n_i)
\end{align}
Electron and ion densities are decompose according to the two time scales
assumed
\begin{align}
n_e & = n_0+\delta n_i + \delta n_e\\
n_i & = n_0+\delta n_i
\end{align}
where $\delta n_i$ is the slow time scale density perturbation (with
quasi-neutrality) and $\delta n_e$ is the fast time scale density
perturbation.
Poisson equation becomes
\begin{align}
\frac{1}{2}\diffp{E}{x}\exp(-i\omega_0t)+\text{c.c.}=
-\frac{e}{\epso}\delta n_e
\end{align}
and in normalised variables
\begin{align}
\tilde{n}(x,t) =
-\frac{1}{\sqrt{3}}\sqrt{\eta}\diffp{\tilde{E}(x,t)}{x}\exp(-i\omega_0t)+
\text{c.c.}
\end{align}
therefore in physical units
\begin{align}
\av{|\delta n(k, \omega)|^2} = \frac{\epso^2}{e^2} k^2\av{|E(k,\omega)|^2}
\end{align}

\section{Stochastic Zakharov equations}
Let use first consider the two following homogeneous wave equations 
for Langmuir waves and ion-acoustic waves
\begin{align}
\diff{E_k}{t}+\left(\nu_e+ik^2\right)E_k &= 0,
\label{eq:lwe}\\
\diff[2]{n_k}{t}+2\nu_i\diff{n_k}{t}+k^2n_k &= 0,
\label{eq:lwn}
\end{align}
and let us recast those deterministic wave equations into their stochastic
counterparts
with complex stochastic complex noise source $W_t$ (Gaussian white noise)
and where $\sigma_E$, $\sigma_n$ and $\sigma_v$ are real valued functions
of the variable $k$. 

These equations can be written as
\begin{align}
\dint{E_k}&=-(\nu_e+ik^2)E_k\dint{t}+\frac{\sigma_E}{\sqrt{2}}
\left(\dint{B_{1}}+i\dint{B_{2}}\right),
\label{eq:sde0}\\
\dint{n_k}&=v_k\dint{t}+\frac{\sigma_n}{\sqrt{2}}
\left(\dint{B_{3}}+i\dint{B_{4}}\right),
\label{eq:sde20}\\
\dint{v_k}&=-2\nu_i v_k\dint{t}-k^2n_k\dint{t}+\frac{\sigma_v}{\sqrt{2}}
\left(\dint{B_{5}}+i\dint{B_{6}}\right),
\label{eq:sde21}
\end{align}
where $B_{2j}$ and $B_{2j{+}1}$ denote Wiener processes describing 
Brownian motion and are given by
\begin{align}
B_{2j}(t) = \int_0^t\Re{W_t}\dint{t},\qquad\text{and}\qquad
B_{2j{+}1}(t) = \int_0^t\Im{W_t}\dint{t}.\nonumber
\end{align}

\subsection{Langmuir wave equation}
Posing 
\begin{align}
\vec{X}\equiv\begin{pmatrix}X_1\\X_2\end{pmatrix}
=\begin{pmatrix}\Re{E_{k}}\\\Im{E_{k}}\end{pmatrix}
\qquad\text{and}\qquad
\dint{\vec{B}}\equiv\begin{pmatrix}\dint{B_1}\\\dint{B_2}\end{pmatrix}.
\end{align}
We can rewrite \eq{eq:sde0} in matrix form as a two-dimensional It\^o
equation
\begin{align}
\dint{\vec{X}}=
\begin{pmatrix}{-}\nu_e&k^2\\{-}k^2&{-}\nu_e\end{pmatrix}
\vec{X}\dint{t}+\frac{\sigma_E}{\sqrt{2}}\dint{\vec{B}},
\end{align}
and making use of the multi-dimensional It\^o-formula \citep{oksendal:2000}, 
we get
\begin{align}
\dint{(\vec{X}^T\vec{X})} = &\;
2X_1\dint{X_1}+2X_2\dint{X_2}+(\dint{X_1})^2+(\dint{X_2})^2\nonumber\\
= &\;2X_1\left(\left(-\nu_eX_1{-}k^2X_2\right)\dint{t}+
\frac{\sigma_E}{\sqrt{2}}\dint{B_{1}}\right)+\nonumber\\
&\;2X_2\left(\left(k^2X_1{-}\nu_eX_2\right)\dint{t}+
\frac{\sigma_E}{\sqrt{2}}\dint{B_{2}}\right)+
\frac{\sigma_E^2}{2}\dint{t}{+}\frac{\sigma_E^2}{2}\dint{t}.\nonumber
\end{align}
Taking the expectation 
$\mathrm{E}[\dint{(\vec{X}^T\vec{X})}]$
and interchanging the integrals,
$\mathrm{E}[\dint{(\vec{X}^T\vec{X})}]\equiv\dint{\av{|E_k(t)|^2}}$
gives the following deterministic differential equation (since the
stochastic terms fall when we take the expectation)
\begin{align}
\dint{\av{|E_k(t)|^2}} &= -2\nu_e\av{|E_k(t)|^2}\dint{t}+\sigma_E^2\dint{t}
\end{align}
which has the analytic solution
\begin{align}
\av{|E_k(t)|^2} = \exp(-2\nu_et)\av{|E_k(0)|^2}+\frac{\sigma_E^2}{2\nu_e}
\left(1-\exp(-2\nu_et)\right)
\end{align}
and finally the stationary (long term) variance is given by
\begin{align}
\lim_{t\rightarrow\infty}\av{|E_k(t)|^2} = \frac{\sigma_E^2}{2\nu_e}
\end{align}

\subsection{Ion acoustic wave equation}
Posing
\begin{align}
\vec{X}\equiv
\begin{pmatrix}X_1\\X_2\\X_3\\X_4\end{pmatrix}=
\begin{pmatrix}\Re{n_{k}}\\\Im{n_{k}}\\\Re{v_{k}}\\\Im{v_{k}}\end{pmatrix}
\qquad\text{and}\qquad
\dint{\vec{B}}\equiv
\begin{pmatrix}\dint{B_3}\\\dint{B_4}\\\dint{B_5}\\\dint{B_6}\end{pmatrix}.
\end{align}
We can rewrite \eqs{eq:sde20}{eq:sde21} in matrix form as a 
four-dimensional It\^o equation
\begin{align}
\dint{\vec{X}}=
\begin{pmatrix}
0&0&1&0\\0&0&0&1\\{-}k^2&0&{-}2\nu_i&0\\0&{-}k^2&0&{-}2\nu_i
\end{pmatrix}
\vec{X}\dint{t}+
\frac{1}{\sqrt{2}}
\begin{pmatrix}
\sigma_n&0&0&0\\0&\sigma_n&0&0\\0&0&\sigma_v&0\\0&0&0&\sigma_v
\end{pmatrix}\dint{\vec{B}},
\end{align}
and making use of the multi-dimensional It\^o-formula, we get
\begin{align}
\dint{X_1^2}+\dint{X_2^2} = &\; 
2X_1\dint{X_1}+2X_2\dint{X_2}+(\dint{X_1})^2+(\dint{X_2})^2\nonumber\\
= &\;
2X_1\left(X_3\dint{t}+\frac{\sigma_n}{\sqrt{2}}\dint{B_3}\right)+
2X_2\left(X_4\dint{t}+\frac{\sigma_n}{\sqrt{2}}\dint{B_4}\right)+
\frac{\sigma_n^2}{2}\dint{t}+\frac{\sigma_n^2}{2}\dint{t}\nonumber\\
\dint{X_3^2}+\dint{X_4^2} = &\; 
2X_3\dint{X_3}+2X_4\dint{X_4}+(\dint{X_3})^2+(\dint{X_4})^2\nonumber\\
= &\; 2X_3 \left(\left(-k^2X_1-2\nu_iX_3\right)\dint{t}+
\frac{\sigma_v}{\sqrt{2}}\dint{B_5}\right)+ \nonumber\\
&\; 2X_4 \left(\left(-k^2X_2-2\nu_iX_4\right)\dint{t}+
\frac{\sigma_v}{\sqrt{2}}\dint{B_6}\right)+
\frac{\sigma_v^2}{2}\dint{t}+\frac{\sigma_v^2}{2}\dint{t}\nonumber
\end{align}
We identify $|n_k|^2=X_1^2+X_2^2$, $|v_k|^2=X_3^2+X_4^2$ and
$n_k\cc{v}_k+\cc{n}_kv_k=2(X_1X_3+X_2X_4)$
and therefore
\begin{align}
\dint{|n_k|^2} &= \left(n_k\cc{v}_k+\cc{n}_kv_k\right)\dint{t}+
\sigma_n^2\dint{t}\\
\dint{|v_k|^2} &= -4\nu_i|v_k|^2\dint{t}-
k^2\left(n_k\cc{v}_k+\cc{n}_kv_k\right)\dint{t}+\sigma_v^2\dint{t}
\end{align}

Taking the expectation and interchanging the integrals
gives (the stochastic terms fall when we take the expectation)
we find
\begin{align}
\dint{\av{|n_k|^2}} &= \av{n_k\cc{v}_k+\cc{n}_kv_k}\dint{t}+
\sigma_n^2\dint{t}\\
\dint{\av{|v_k|^2}}+4\nu_i\av{|v_k|^2}\dint{t} &= -
k^2\av{n_k\cc{v}_k+\cc{n}_kv_k}\dint{t}+\sigma_v^2\dint{t}
\end{align}

Let use estimate a differential equation for the expectation 
$\av{n_k\cc{v}_k+\cc{n}_kv_k}$. Making use of the multi-dimensional
It\^o-formula, we get
\begin{align}
\dint{X_1X_3}=&\;X_1\dint{X_3}+X_3\dint{X_1}+\dint{X_1}\dint{X_3}\nonumber\\
= &\;
X_1\left(\left(-k^2X_1-2\nu_iX_3\right)\dint{t}+
\frac{\sigma_v}{\sqrt{2}}\dint{B_5}\right)+
X_3\left(X_3\dint{t}+\frac{\sigma_n}{\sqrt{2}}\dint{B_3}\right)+\nonumber\\
&\; \left(X_3\dint{t}+\frac{\sigma_n}{\sqrt{2}}\dint{B_3}\right)
\left(\left(-k^2X_1-2\nu_iX_3\right)\dint{t}+
\frac{\sigma_v}{\sqrt{2}}\dint{B_5}\right)\nonumber\\
\dint{X_2X_4}=&\;X_2\dint{X_4}+X_4\dint{X_2}+\dint{X_2}\dint{X_4}\nonumber\\
= &\;
X_2\left(\left(-k^2X_2-2\nu_iX_4\right)\dint{t}+
\frac{\sigma_v}{\sqrt{2}}\dint{B_6}\right)+
X_4\left(X_4\dint{t}+\frac{\sigma_n}{\sqrt{2}}\dint{B_4}\right)+\nonumber\\
= &\;\left(X_4\dint{t}+\frac{\sigma_n}{\sqrt{2}}\dint{B_4}\right)
\left(\left(-k^2X_2-2\nu_iX_4\right)\dint{t}+
\frac{\sigma_v}{\sqrt{2}}\dint{B_6}\right)\nonumber
\end{align}
and
\begin{align}
\dint{\av{X_1X_3}}=-k^2\av{X_1^2}\dint{t}-2\nu_i\av{X_1X_3}\dint{t}+
\av{X_3^2}\dint{t}\nonumber\\
\dint{\av{X_2X_4}}=-k^2\av{X_2^2}\dint{t}-2\nu_i\av{X_2X_4}\dint{t}+
\av{X_4^2}\dint{t}\nonumber
\end{align}
which can be rewritten as
\begin{align}
\dint{\av{n_k\cc{v}_k+\cc{n}_kv_k}}+2\nu_i\av{n_k\cc{v}_k+\cc{n}_kv_k}\dint{t}
=-2k^2\av{|n_k|^2}\dint{t}+2\av{|v_k|^2}\dint{t}
\end{align}
Posing $x=\av{|n_k|^2}$, $y=\av{|v_k|^2}$ and
$z=\av{n_k\cc{v}_k+\cc{n}_kv_k}$, $x$, $y$ and $z$ are solution of the
following system of ordinary differential equations
\begin{align}
\diff{x}{t}&=z+\sigma_n^2\\
\diff{y}{t}&=-4\nu_iy-k^2z+\sigma_v^2\\
\diff{z}{t}&=-2k^2x+2y-2\nu_iz
\end{align}
together with the initial conditions
\begin{align}
x(0)&=\av{|n_k(0)|^2}\\
y(0)&=\av{|v_k(0)|^2}\\
z(0)&=\av{n_k(0)\cc{v}_k(0)+\cc{n}_k(0)v_k(0)}
\end{align}

This system of ordinary differential equations have solution so that posing
\begin{align}
X=\begin{pmatrix}x\\y\\z\end{pmatrix}\qquad\text{and}\qquad
A=\begin{pmatrix}0&0&1\\0&{-}4\nu_i&{-}k^2\\{-}2k^2&2&-2\nu_i\end{pmatrix}
\qquad\text{and}\qquad
b=\begin{pmatrix}\sigma_n^2\\\sigma_v^2\\0\end{pmatrix}\nonumber
\end{align}
we can formulate the problem as
\begin{align}
\diff{X}{t}=AX+b
\end{align}
The eigen values $\lambda_i$ and eigen vectors $g_i$ of $A$ can be calculated
analytically and are
\begin{align}
\lambda_1 &= -2\nu_i+2\sqrt{\nu_i^2-k^2}, &
g_1 & = \begin{pmatrix}1\\
k^2-2\sqrt{\nu_i^2-k^2}\left(\nu_i-\sqrt{\nu_i^2-k^2}\right)\\
-2\left(\nu_i-\sqrt{\nu_i^2-k^2}\right)\end{pmatrix}\\
\lambda_2 &= -2\nu_i, &
g_2 & = \begin{pmatrix}1\\k^2\\-2\nu_i\end{pmatrix}\\
\lambda_3 &= -2\nu_i-2\sqrt{\nu_i^2-k^2}, &
g_3 & = \begin{pmatrix}1\\
k^2+2\sqrt{\nu_i^2-k^2}\left(\nu_i+\sqrt{\nu_i^2-k^2}\right)\\
-2\left(\nu_i+\sqrt{\nu_i^2-k^2}\right)\end{pmatrix}
\end{align}
In the same way the quantity
\begin{align}
A^{-1} b = \begin{pmatrix}
-\frac{1}{4\nu_ik^2}\left(\sigma_v^2+k^2\sigma_n^2\right)-
\frac{\nu_i}{k^2}\sigma_n^2\\
-\frac{1}{4\nu_ik^2}\left(\sigma_v^2+k^2\sigma_n^2\right)\\
\sigma_n^2
\end{pmatrix}
\end{align}

\subsubsection{$k\neq0$}
The solution takes the following form
\begin{align}
X(t)&=-A^{-1}b+ c_1\exp(\lambda_1t)g_1+ c_2\exp(\lambda_2t)g_2+
c_3\exp(\lambda_3t)g_3\\
X(0)&=-A^{-1}b+ c_1g_1+ c_2g_2+c_3g_3\label{eq:csts}
\end{align}
where the constants $(c_1,c_2,c_3)$ are calculated using \eq{eq:csts}.
Since all the eigen values have negative real part we have
\begin{align}
\lim_{t\rightarrow\infty} X(t) = -A^{-1} b
\end{align}
and therefore
\begin{align}
\lim_{t\rightarrow\infty} \av{|n_k(t)|^2} = 
\frac{1}{4\nu_ik^2}\left(\sigma_v^2+k^2\sigma_n^2\right)+
\frac{\nu_i}{k^2}\sigma_n^2 = 
\left(\frac{1}{4\nu_i}+\frac{\nu_i}{k^2}\right)\sigma_n^2+
\frac{\sigma_v^2}{4\nu_ik^2}
\end{align}
\subsubsection{$k=0$}
In this case we can also integrate analytically to get
\begin{align}
x(t)&=\frac{y(0)}{\nu_i}\frac{e^{-4\nu_it}{-}1}{4\nu_i}+
\frac{\sigma_v^2}{4\nu_i^2}
\left(t+\frac{e^{-4\nu_it}{-}1}{4\nu_i}\right)+
\left(z(0){+}\frac{y(0)}{\nu_i}{-}
\frac{\sigma_v^2}{2\nu_i^2}\right)\frac{e^{-2\nu_it}{-}1}{2\nu_i}
+\sigma_n^2t+x(0)\nonumber\\
y(t) &= y(0)e^{-4\nu_it}+
\frac{\sigma_v^2}{4\nu_i}\left(1{-}e^{-4\nu_it}\right)\nonumber\\
z(t) &= -\frac{y(0)}{\nu_i}e^{-4\nu_it}+
\frac{\sigma_v^2}{4\nu_i^2}\left(1+e^{-4\nu_it}\right)+
\left(z(0)+\frac{y(0)}{\nu_i}-\frac{\sigma_v^2}{2\nu_i^2}\right)e^{-2\nu_it}
\nonumber
\end{align}
and therefore the asymptotic behaviour
\begin{align}
\lim_{t\rightarrow\infty} \av{|n_k(t)|^2} =
\left(\sigma_n^2+\frac{\sigma_v^2}{4\nu_i^2}\right)t
\end{align}

\subsection{Ornstein-Uhlenbeck process}

The Ornstein-Uhlenbeck process also known as the mean reverting
process in financial modelling, is a stochastic process $S_t$ defined by the
following stochastic differential equation
\begin{align}
\dint{S_t}&=\theta(\mu-S_t)\dint{t} + \sigma \dint{W_t}
\end{align}
where $\theta>0$, $\mu$ and $\sigma>0$ are parameters and $W_t$ denotes the
Wiener process. $1/\theta$ and $\sigma$ are the relaxation time and the
diffusion constant. $\theta$, $\mu$ and $\sigma$ are called mean reversion rate,
the mean and the volatility in financial mathematics. 
The stationary long term mean, variance and covariance are given by
\begin{align}
\av{S_t} & = \mu, \\
\av{S_t^2} &= \frac{\sigma^2}{2\theta}, \\
\av{S_sS_t} &= \frac{\sigma^2}{2\theta}e^{-\theta(s{+}t)}
\left(e^{-\theta(t{-}s)}{-}e^{-\theta(t{+}s)}\right)\quad\text{for}\quad s<t,
\end{align}
The Ornstein-Uhlenbeck stochastic process has an exact solution which allows
to simulate it numerically with arbitrary time steps given an initial value
$S(0)$ \citep{gillespie:1996c,gillespie:1996b}
\begin{align}
S(t) &= S(0)e^{-\theta t} + 
\mu\left(1-e^{-\theta t}\right) + 
\frac{\sigma}{\sqrt{2\theta}}\sqrt{1-e^{-2\theta t}}N(0,1).
\end{align}
Equivalently, it can be seen as the a normal distributed with mean 
$S(0)\exp\left(-\lambda t\right)+\mu\left(1-\exp\left(-\lambda t\right)\right)$
and standard deviation 
$\sigma\sqrt{\frac{1-\exp\left(-2\lambda t\right)}{2\lambda}}$

\section{Noise source terms for Cerenkov emission}

$\dnk(t)$ and $\dEk(t)$ are noise source terms accounting
for Cerenkov emission. Their form is such that thermal equilibrium is
regained in the absence of an external drive. Explicit expressions
are for the slow time scale equation
\begin{align}
\left|\frac{\dnk}{n_e}\right|^2&=\frac{1}{N}\frac{1}
{(1+k^2\lambda_e^2)(1+\Te/T_i+k^2\lambda_e^2)}
\end{align}
while for the fast time scale equation
\begin{align}
\left|\frac{\dnk}{n_e}\right|^2&=\frac{1}{N}\frac{1}{2}\frac{k^2\lambda_e^2}
{1+k^2\lambda_e^2}
\end{align}
which gives for the fluctuating electric field
\begin{align}
\frac{\varepsilon_0|\dEk|^2}{n_e\kB
\Te}&=\frac{1}{N}\frac{1}{2}\frac{1}{1+k^2\lambda_e^2}
\end{align}


$\dnk(t)$ and $\dEk(t)$ are noise source terms accounting
for Cerenkov emission. Their form is such that thermal equilibrium is
regained in the absence of an external drive. Explicit expressions
are as follow
\begin{align}
\left|\frac{\dnk}{n_e}\right|^2&=\frac{1}{N}\frac{1+\overline{\chi}_e(k)}
{\overline{\epsilon}(k,0)},\nonumber\\
&=\frac{1}{N}\frac{1+k^2\lambda_e^2}{1+k^2\lambda_e^2+0.5\Te/\Ti}.\\
\frac{\varepsilon_0|\dEk|^2}{n_e\kB\Te}&=\frac{1}{N}\frac{{\chi}_e(k,0)}{
1+{\chi}_e(k,0)}+\frac{1}{N}\frac{{\chi}_e(k,0)
\left(1+\overline{\chi}_e(k)\right)}{
\overline{\epsilon}(k,0)\left(1+{\chi}_e(k,0)\right)},\nonumber\\
&=\frac{1}{N}\frac{1}{2k^2\lambda_e^2+1}+\frac{1}{N}
\frac{(k^2\lambda_e^2+1)k^2\lambda_e^2}{2(k^2\lambda_e^2+1+0.5\Te/\Ti)
(k^2\lambda_e^2+0.5)^2}
\end{align}

\section{Phase-space diffusion}

The phase-space diffusion equation reads
\citep{sanbonmatsu:2000b,sanbonmatsu:2001}
\begin{align}
\diffp{}{t}F_e(v,t) = \diffp{}{v}\left(D(v,t)\diffp{}{v}F_e(v,t)\right)
%+\diffp{}{v}\delta D(v,t)F_e(v,t)/v_e^2
-\frac{v}{L}\left(F_e(v,t)-F_e(v,t{=}0)\right)
\end{align}
where $F_e$ is the spatially averaged over the length $L$ of the 
electron velocity distribution. 
The diffusion coefficient $D$  is of the standard form 
\begin{align}
D(v,t) = \frac{L}{|v|}\left(\frac{e}{m}\right)^2\left|E_{k=\omega_r/|v|}(t)\right|^2
\end{align}
For convenience we change the variable $v$ into $k=\omega_r/v$ which yields
\begin{align}
\diffp{}{t}F_e(k,t) = \diffp{}{k}\left(
D(k,t)\diffp{}{k}F_e(k,t)\diffp{k}{v}\right)\diffp{k}{v}
%+\diffp{}{v}\delta D(v,t)F_e(v,t)/v_e^2
-\frac{\omega_r}{kL}\left(F_e(k,t)-F_e(k,t=0)\right)
\end{align}
where 
\begin{align}
\diffp{k}{v}=-\frac{k|k|}{\omega_e^2}\omega_r
\end{align}
This equation can be differenced over a grid $k_j$ at times $n$
using the Crank-Nicholson method,  i.e.
\begin{align}
\frac{F_j^{n+1}-F_j^n}{\Delta t}  = & 
\frac{D_{j{+}\frac{1}{2}}^n\left(F_{j+1}^n-F_j^n\right)
\partial_vk_{j{+}\frac{1}{2}}-
D_{j{-}\frac{1}{2}}^n\left(F_j^n-F_{j-1}^n\right)\partial_vk_{j{-}\frac{1}{2}}}
{2(\Delta k)^2}\partial_vk_j+\\
& \frac{D_{j{+}\frac{1}{2}}^n\left(F_{j+1}^{n+1}-F_j^{n+1}\right)
\partial_vk_{j{+}\frac{1}{2}}-
D_{j{-}\frac{1}{2}}^n\left(F_j^{n+1}-F_{j-1}^{n+1}\right)
\partial_vk_{j{-}\frac{1}{2}}}{2(\Delta k)^2}\partial_vk_j\\
& -\frac{\omega_r}{k_jL}\left(\frac{F_j^n-F_j^0}{2}+\frac{F_j^{n+1}-F_j^0}{2}\right)
\end{align}
where
\begin{align}
D_{j{+}\frac{1}{2}}^n = \frac{1}{2}\left(D_j^n+D_{j+1}^n\right) 
\qquad\text{and}\qquad
\partial_vk_{j{+}\frac{1}{2}} = -\frac{\omega_e^2}{\omega_r}
\left(\frac{k_j|k_j|}{2}+\frac{k_{j{+}1}|k_{j{+}1}|}{2}\right)
\end{align}
and the following boundary conditions
\begin{align}
F_0^{n+1}=F_0^n  \qquad\text{and}\qquad 
F_N^{n+1}=F_N^n 
\end{align}
Defining 
\begin{align}
\alpha_j & = -\frac{\Delta t}{2(\Delta k)^2}\partial_vk_{j{-}\frac{1}{2}}D_{j{-}\frac{1}{2}}^n
\partial_vk_j,\\
\gamma_j & = -\frac{\Delta t}{2(\Delta k)^2}\partial_vk_{j{+}\frac{1}{2}}D_{j{+}\frac{1}{2}}^n
\partial_vk_j,\\
\beta_j & = 1 -\alpha_j-\gamma_j-\epsilon_j\\
\delta_j & = 1 +\alpha_j+\gamma_j+\epsilon_j\\
\epsilon_j & = -\frac{\omega_r\Delta t}{2k_jL}
\end{align}
Reorganising the terms leads to 
\begin{align}
F_0^{n+1}& =F_0^n\\
\alpha_jF_{j-1}^{n+1}+\beta_jF_j^{n+1}+\gamma_jF_{j+1}^{n+1} & = 
-\alpha_jF_{j-1}^n+\delta_jF_j^n-
\gamma_jF_{j+1}^n+2\epsilon_jF_j^0\\
F_N^{n+1}&=F_N^n
\end{align}
Using the tridiagonal matrix $T$
\begin{align}
\mathrm{Tri}(a,b,c)=
\begin{pmatrix}
1&0&0&0&\ldots&0&0&0&0\\
a_1&b_1&c_1&0&\ldots&0&0&0&0\\
0&\ddots&\ddots&\ddots&&&&&\vdots\\
\vdots&&\ddots&\ddots&\ddots&&&&\vdots\\
0&0&0&a_j&b_j&c_j&0&0&0\\
\vdots&&&&\ddots&\ddots&\ddots&&\vdots\\
\vdots&&&&&\ddots&\ddots&\ddots&0\\
0&0&0&0&\ldots&0&a_{N{-}1}&b_{N{-}1}&c_{N{-}1}\\
0&0&0&0&\ldots&0&0&0&1
\end{pmatrix}
\end{align}
the set of equation is written
\begin{align}
\mathrm{Tri}(\alpha,1{-}\alpha{-}\gamma{-}\epsilon,\gamma)\vec{F}^{n+1} =
\mathrm{Tri}(-\alpha,1{+}\alpha{+}\gamma{+}\epsilon,-\gamma)\vec{F}^{n} +
\mathrm{Tri}(0,2\epsilon,0)\vec{F}^{0}
\end{align}
This system can be inverted using a tridiagonal algorithm

The quasi-linear time $\tau_{1/2}$ which corresponds to the time to create
half-plateau down to half of the beam width $v_b/2$ is written \citep{kontar:2001}
\begin{align}
\tau_{1/2} = \frac{n_e}{n_b}\frac{1}{\omega_e},
\end{align}
in normalised units
\begin{align}
\tilde{\tau}_{1/2} = \frac{n_e}{n_b}\frac{2}{3}\frac{\eta m_e}{m_i}.
\end{align}



\bibliography{abbrevs,research,books}

\end{document}
